\chapter{はじめに}
 % デフォルトの箇条書きは項目間や段落間のスペースが広いので下記のように調整した方が綺麗に見えるかも
 \setlength{\parskip}{0cm} % 段落間
 \setlength{\itemsep}{0cm} % 項目間

    近年, 生化学分野において,実験室規模で行われていた従来の実験方法に代わる新たな実験装置としてバイオチップが注目されている~\cite{10.1146}\cite{urbanski2006}\cite{gubala2012point}.
    バイオチップの一種であるPMD(Programmable Microfluidic Device)は\bout{特に}注目されている~\cite{1}\cite{9045675}\cite{10247720}.
    PMDは,バルブと,バルブに囲まれたセルによって構成されるデバイスである.
    PMDは,バルブの開閉を行うことによりミキサーと呼ばれる環状の流路を作り,液滴の混合を行う~\cite{FU2015343}\cite{7926964}.

    ミキサーを用いた液滴の混合を複数回行うことで,PMDは複数種類の試薬を\bout{目標の}比率で混合する.
    この処理のことを試薬合成と呼ぶ~\cite{2}\cite{poddar2021generic}.
    PMD上での試薬合成中にはミキサーを用いた液滴の混合によって生成された中間液滴が\rout{,}他のミキサーの使用するセルに残され,液滴の混合を進められない状態が発生する場合がある.
    この状態のことをミキサーのオーバーラップと呼ぶ.
    ミキサーのオーバーラップには,残された中間液滴を緩衝液という液体で洗い流す,フラッシングという操作を行うことで対処する~\cite{8715125}.
    試薬合成中に行われるフラッシング回数と比例して,試薬合成における使用緩衝液量は増加する.
    使用緩衝液量の増加は実験コストの増加に繋がる.したがって,実験コストの増加を抑えるために試薬合成中のフラッシング回数を削減することが求められている.

    PMD上での試薬合成に関連した様々な手法が提案されている~\cite{bhattacharjee2022algorithms}.
    PMD上での試薬合成における液滴の混合手順の生成手法としては,No Transport Mixing(NTM)が提案されている~\cite{4}.
    また,試薬合成中に行われるフラッシング回数を削減する手法としては,\bout{配置優先度}を用いた平井らの手法が提案されている~\cite{10089903}.
    本論文では,PMD上での試薬合成におけるフラッシング回数の削減手法,スケーリングを新たに提案する.
    \bout{本論文で提案する手法のスケーリングでは,他のミキサーとのオーバーラップが発生する確率が高いと判断した全てのミキサーの大きさが2倍になるように混合木を変形する.大きさを2倍にされたミキサーは,配置に使用するPMDのセル数が2倍になる.したがって,配置方法に様々な選択肢を持つことが可能となり,他のミキサーとのオーバーラップが発生する配置方法を避けることが容易になる.}
    %前述した通り,ミキサーのオーバーラップの発生回数が増加するほど,試薬合成に\rout{おける}フラッシング回数も増加する.
    %したがって,\rout{試薬合成における}フラッシング回数を\rout{削減する}ためには,ミキサーのオーバーラップの発生回数を\rout{削減すれ}ば良い.
    %それを踏まえて,本論文で提案する手法のスケーリングでは,試薬合成の入力データである混合木を変形することによって,試薬合成におけるミキサーのオーバーラップの\rout{発生}回数を削減する.
    %これによって,\rout{本手法は}試薬合成\rout{における}フラッシング回数を削減する.
    \rout{\bout{平井らの手法で}は,配置優先度と呼ばれる指標を用いて,試薬合成中の液滴の混合順序を並び替えることによってフラッシング回数の削減を行っていた.}
    \rout{それに対して,\mout{提案手法}では\bout{平井らの手法}と異なる\bout{指標を用いて}試薬合成中の液滴の混合順序の並び替えを行うのに加えて,混合木の変形も行う.これにより,液滴の混合順序の並び替えだけでは削減することの出来なかったフラッシングも,\mout{提案手法}では削減することが可能になる.}

    実験では,\bout{平井らの手法}と提案手法のそれぞれを用いた試薬合成において必要となるフラッシング回数を比較した.
    その結果,提案手法はPP Valueよりも70\%少ないフラッシング回数での試薬合成が可能であることを確認した.
    
    本論文は\rout{6}章で構成されている.
    第2章では,PMDを用いた試薬合成に関する基礎知識について述べる.
    第3章では,\bout{配置優先度}を用いた試薬合成中に行われるフラッシング回数の削減手法について述べる.
    第4章では,スケーリングを用いた試薬合成中に行われるフラッシング回数の削減手法について述べる.
    第5章では,実験結果と考察について述べる.
    第6章では,本論文のまとめと,今後の課題について述べる.

