\chapter{はじめに}
\begin{itemize}
 % デフォルトの箇条書きは項目間や段落間のスペースが広いので下記のように調整した方が綺麗に見えるかも
 \setlength{\parskip}{0cm} % 段落間
 \setlength{\itemsep}{0cm} % 項目間
 \item どのような分野の研究か,その背景について説明する.
    \begin{itemize}
        \item PMDとは,試薬合成に用いられるバイオチップの一種である.
        \item 試薬合成中に行われるフラッシングの回数と比例して,試薬合成において使用される緩衝液の量は増加する..
        \item 実験コストを抑えるために,試薬合成中のフラッシング回数を減らすことが求められている.
    \end{itemize}
 \item その分野の従来の研究状況について説明する.
     \begin{itemize}
        \item PMDを用いた試薬合成手法では,NTMが提案されている.
        \item 試薬合成中に行われるフラッシング回数を削減する手法としては,PP valueを用いた手法が提案されている.
    \end{itemize}

 \item そして,何が解決すべき問題(本論文で扱った問題)かを説明.
    \begin{itemize}
        \item フラッシング回数を削減する.
    \end{itemize}
 \item どのようなアイデアで解決したか,キーアイデアを少しだけ披露
    \begin{itemize}
        \item 入力データである混合木を変形する,スケーリングという手法を提案し,PP Valueと併用する.

    \end{itemize}
 \item どのような(実験)結果が得られたか、アピール(目次案の段階では希望的予測)
    \begin{itemize}
        \item スケーリングを用いた場合と用いなかった場合のフラッシング回数を比較する.
        \item スケーリングにより,フラッシング回数が50\%削減された.
    \end{itemize}
\item 章構成
    \begin{itemize}
        \item 本論文は5章で構成されている.
        \item 第2章では,PMDを用いた試薬合成に関する基礎知識について述べる.
        \item 第3章では,PP valueを用いた試薬合成中に行われるフラッシング回数の削減手法について述べる.
        \item 第4章では,スケーリングを用いた試薬合成中に行われるフラッシング回数の削減手法について述べる.
        \item 第5章では,実験結果と考察について述べる.
        \item 第6章では,本論文のまとめと,今後の課題について述べる.
    \end{itemize}
\end{itemize}

