\chapter{スケーリングを用いた試薬合成中におけるフラッシング回数の削減手法}
本章では,提案手法について説明する.

\section{スケーリングを用いた入力混合木の変形アルゴリズム}
\subsection{スケーリングの概要}
混合木の図などを用いてスケーリングという操作について説明を行う.
\subsection{試薬合成中に行われるフラッシングの回数を増加させる提供比率}
\begin{itemize}
\item 内部フラッシング(IF)の説明と,IFを起こす確率に基づく混合木内の提供比率のグループ分けの説明\cite{1}\cite{2}.
\item 混合木を入力として受け取った直後に,その混合木に含まれるIFが必要となる可能性の高い提供比率のx2スケーリングを行う.
\end{itemize}
\section{PMD上での大きなミキサーを用いた液滴の混合手順生成アルゴリズム}
\subsection{使用セル数推定値を用いた試薬合成における液滴の混合順序の\\並び替えアルゴリズム}
\begin{itemize}
\item ミキサーを根とした部分木の使用セル数を推定した値の順番に,試薬合成における液滴の混合順序を並び替える.
\end{itemize}
\subsection{ミキサーの配置方法決定アルゴリズム}
\begin{itemize}
\item ミキサーの配置方法決定アルゴリズムについて,疑似コードを踏まえながら説明する.
\item 液滴の混合手順の生成処理中にミキサーのIF無しでの配置方法が見つからない場合,該当提供比率のx2スケーリングを行う.
\end{itemize}
