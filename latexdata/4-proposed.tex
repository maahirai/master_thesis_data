\chapter{スケーリングを用いた試薬合成中におけるフラッシングの回数の削減手法}
本章では,提案手法について説明する.

\section{スケーリングを用いた入力混合木の変形アルゴリズム}
\subsection{スケーリングの概要}
混合木の図などを用いてスケーリングという操作について説明を行う.
\subsection{試薬合成中に行われるフラッシングの回数を増加させる提供比率}
\begin{itemize}
\item 内部フラッシング(IF)の説明と,IFを起こす確率に基づく混合木内の提供比率のグループ分けの説明\cite{1}\cite{2}.
\item 混合木を入力として受け取った直後に,その混合木に含まれるIFが必要となる可能性の高い提供比率のx2スケーリングを行う.
\end{itemize}

\section{PMD上での大きなミキサーを用いた液滴の混合手順生成アルゴリズム}
\begin{itemize}
\item 液滴の混合手順生成(ミキサーレイアウト決定)アルゴリズムについて,疑似コードを踏まえながら説明する.
\item 液滴の混合手順の生成処理中にミキサーのレイアウト方法が見つからない場合,該当提供比率のx2スケーリングを行う.
\end{itemize}
