\chapter{スケーリングを用いた試薬合成中におけるフラッシング回数の削減手法}
\section{\mout{アルゴリズムの概要}}
本論文の提案手法である,スケーリングを用いた試薬合成中におけるフラッシング回数の削減手法は,スケーリングという処理を行うことで入力された混合木の変形を行い,試薬合成中に行われるフラッシングの回数を削減する手法である.
本節では,スケーリングを用いた試薬合成中におけるフラッシング回数の削減手法の処理の流れを説明する.

提案手法での手法全体の入出力データについて説明を行う.
提案手法の入力は,図~\ref{fig:ScalingInputOutput}(a)の混合木である.この混合木は,図~\ref{fig:xntm}(a)と同じ混合木である.
提案手法の出力は,図~\ref{fig:ScalingInputOutput}(b)から(d)のPMD上での試薬合成における液滴の混合手順である.
図~\ref{fig:ScalingInputOutput}(b)から(d)における値Tはミキサーの混合が行われたタイムステップ,値Fはそのタイムステップに至るまでに行ったフラッシング回数を表している.

平井らの手法では,図~\ref{fig:xntm}で示した通り,図~\ref{fig:ScalingInputOutput}(a)の混合木の試薬合成に1回のフラッシングを必要とする.
それに対して,提案手法は図~\ref{fig:ScalingInputOutput}(a)の混合木の試薬合成をフラッシングなしで行うことが可能である.
したがって,図~\ref{fig:ScalingInputOutput}(a)の混合木の試薬合成においては,提案手法は平井らの手法よりフラッシングを削減することに成功している.

Algorithm~\ref{alg:allscaling}に,提案手法全体の処理の流れの疑似コードを示した.
提案手法を大きく分けると,Algorithm~\ref{alg:allscaling}の\ref{alg:scaling_pseudo}~行目のスケーリングを用いた入力混合木の変形と,Algorithm~\ref{alg:allscaling}の\ref{alg:samplepreparation_pseudo}~行目のPMD上での大きなミキサーを用いた液滴の混合手順生成の2つの処理によって構成されている.

\begin{figure}[tbp]
 \centering\includegraphics[scale=0.51]{img/ScalingInputOutput.pdf}
 \caption{提案手法の入力:混合木,出力:PMD上での液滴の混合手順}\label{fig:ScalingInputOutput}
\end{figure}

\begin{algorithm}[tbp]
 \caption{提案手法の処理の流れ}\label{alg:allscaling}
 \begin{algorithmic}[1]
     \Require $\mathit{Tree}$:2$\times$2ミキサーノードと2$\times$3ミキサーノード,試薬液滴ノードを含む混合木
     \Require $\mathit{PMDSize}$:使用するPMDのサイズ
     \State $\mathit{TransformedTree} \gets$ \Call{Scaling}{$Tree$} \Comment{スケーリングを用いた入力混合木の変形}\label{alg:scaling_pseudo}
     \State $\mathit{MixInfo \gets}$\Call{SamplePreparation}{$\mathit{Tree,PMDSize}$} \Comment{混合手順の生成} \label{alg:samplepreparation_pseudo}

      \Return $\mathit{MixInfo}$
 \end{algorithmic}
\end{algorithm}

\section{スケーリングを用いた入力混合木の変形アルゴリズム}
本節では,Algorithm~\ref{alg:allscaling}の\ref{alg:scaling_pseudo}~行目のスケーリングを用いた入力混合木の変形処理について説明する.
%混合木の図などを用いてスケーリングという操作について説明を行う.
\subsection{スケーリングの概要}
%図~\ref{fig:Scaling}(a)に,スケーリングを用いた入力混合木の変形を行う前の混合木,図~\ref{fig:Scaling}(b)に,スケーリングを用いた入力混合木の変形後の混合木を示した.
スケーリングは,混合木内の任意の提供比率を2倍にする操作である.
図~\ref{fig:Scaling}(a)の混合木において,M5への提供比率に対するスケーリングを行った場合,図~\ref{fig:Scaling}(b)の混合木になる.
M5への提供比率に対するスケーリングによって,図~\ref{fig:Scaling}(a)の混合木では3:1だったM5への提供比率は,図~\ref{fig:Scaling}(b)の混合木では2倍されて6:2となっている.
また,スケーリングによって,M5のミキサーサイズは4から8へと変化している.

図~\ref{fig:ScalingInDetail}は,図~\ref{fig:Scaling}におけるM5への提供比率に対するスケーリングの過程を詳細に示した図である.
まず,図~\ref{fig:ScalingInDetail}(a),(b)間では,スケーリングの対象となるM5への提供比率を2倍にしている.
その結果,図~\ref{fig:ScalingInDetail}(b)の混合木のように,混合木内に矛盾点が生じる場合がある.
図~\ref{fig:ScalingInDetail}(b)の混合木における矛盾点とは,ミキサーサイズ4の混合ノードM9がM5への提供比率において6個の中間液滴を提供することになっているというものだ.
自身のサイズ以上の体積の中間液滴を,ミキサーは提供することはできない.
M9のミキサーが,6個の中間液滴を親ノードであるM5のミキサーに提供するためには,ミキサーサイズを6より大きくする必要がある.
したがって,図~\ref{fig:ScalingInDetail}(b),(c)間ではM9への提供比率もスケーリングによって2倍にすることで,M9のミキサーサイズを8にしている.
M9への提供比率に対する操作によって,混合木内の矛盾点は無くなる.
スケーリング後の図~\ref{fig:ScalingInDetail}(c)は,図~\ref{fig:Scaling}(b)と同一の混合木である.

\begin{figure}[tbp]
 \centering\includegraphics[scale=0.7]{img/Scaling.pdf}
 \caption{スケーリングを用いた入力混合木の変形}\label{fig:Scaling}
\end{figure}

\begin{figure}[tbp]
 \centering\includegraphics[scale=0.6]{img/ScalingInDetail.pdf}
 \caption{スケーリングを用いた入力混合木の変形の詳細}\label{fig:ScalingInDetail}
\end{figure}

\begin{figure}[tbp]
 \centering\includegraphics[scale=0.44]{img/merge.pdf}
 \caption{スケーリングの際に発生する,混合ノードのマージ}\label{fig:Merge}
\end{figure}

スケーリング後には,混合ノードのマージを行う場合がある.
スケーリング後に混合ノードのマージを行う混合木の例を,図~\ref{fig:Merge}に示す.
図~\ref{fig:Merge}では,M0への提供比率に対するスケーリングを行う.
スケーリングによって,図~\ref{fig:Merge}(a)の混合木において2:1:1だったM0への提供比率は,図~\ref{fig:Merge}(b)の混合木において4:2:2になっている.
図~\ref{fig:Merge}(b)の混合木において,ミキサーサイズ4のM1のミキサーは,混合した4個全ての中間液滴を親ノードのミキサーに提供することになっている.
混合した全ての中間液滴を親ノードのミキサーへ提供する,M1のミキサーでの液滴の混合は無意味であるため,省略可能である.
したがって,混合ノードM1を削除することで,図~\ref{fig:Merge}(b)の混合木には図~\ref{fig:Merge}(c)の混合木への変形が行われる.
このように,スケーリングに伴って,省略可能になった混合ノードを混合木から削除する操作のことを,混合ノードのマージという.

\subsection{試薬合成中に行われるフラッシングの回数を増加させる提供比率}\label{ratio}
\ref{sec:flushing}~節では,各ノードの提供比率など混合木の情報から,ミキサーのオーバーラップの発生を予測することがある程度可能であることについて述べた.
本節では,提供比率とミキサーのオーバーラップの関係性と,試薬合成中に行われるフラッシング回数を削減するためのスケーリングの活用方法について述べる.

表~\ref{table:ProvRatioGroup}に,試薬合成中にミキサーのオーバーラップを発生させる確率に基づいて分けられた,提供比率のグループを示す.
上部のグループに属する提供比率ほど,ミキサーのオーバーラップを発生させる確率が高い.
最もミキサーのオーバーラップを発生させる確率が高いグループAは,図~\ref{fig:flushing}(a)の点線部の5:1のような,5などの3以上の奇数個の液滴を提供する混合ノードが含まれる提供比率である.
\ref{sec:flushing}~節で説明した通り,グループAの提供比率はミキサーのオーバーラップを発生させることが確定しており,オーバーラップ確定提供比率と呼ばれる.
提供比率がグループAに属するか否かの判定には,提供比率に含まれる混合ノード数は関係ないので,混合ノード数の条件は*(ドントケア)とした.

あるノードの子ノードのミキサーは,自身の親ノードのミキサーの配置セルと重なるように配置される.
したがって,特に親ノードのミキサーサイズが4や6など小さい場合,それぞれの子ノードのミキサーの配置セルは近接しやすい.
ゆえに,提供比率に含まれる混合ノード数が多いほど,兄弟ノード同士でのミキサーのオーバーラップが発生しやすいといえる.
したがって,グループBとグループCの提供比率は,提供比率に含まれる混合ノード数で区分する.
2番目にミキサーのオーバーラップを発生させる確率が高いグループBは,提供比率に含まれる混合ノード数が3個以上であることが所属の条件となる.
また,最もミキサーのオーバーラップを発生させる確率が低いグループCは,提供比率に含まれる混合ノード数が0個から2個であることが所属の条件となる.

本節で説明した通り,スケーリングは混合ノードの提供比率を変更する.
提案手法は,スケーリングを行うことで,グループAの提供比率をグループBかグループCの提供比率に変更する.
提案手法では,スケーリングを行うことで,混合木内のグループAの提供比率の数を削減する.

また,スケーリング後の提供比率に含まれる混合ノード数が増加しない場合,提案手法はグループBの提供比率に対してもスケーリングを行う.
スケーリングは,液滴の提供を受ける親ノードのミキサーサイズを増大させる.
親ノードのミキサーが大きいほど,子ノードのミキサーの配置方法の選択肢は増大する.
したがって,スケーリングを行うことによって子ノードのミキサーの配置方法の選択肢が増え,ミキサーのオーバーラップが発生しない,ミキサーの配置方法の選択が容易になる.

以上で述べた通り,提案手法は,グループAとグループBの提供比率に対してスケーリングを行うことで,よりミキサーのオーバーラップを発生させにくい提供比率に変更する.
提案手法によって,ミキサーのオーバーラップが発生しにくくなり,試薬合成におけるフラッシング回数は削減される.

\begin{table}[tbp]
\centering
    \caption{ミキサーのオーバーラップを発生させる確率の高さに基づいて分けられた,提供比率のグループ}
\begin{tabular}{l|r|r} \Hline
    &\multicolumn{1}{l|}{混合ノード数}& \multicolumn{1}{l}{備考} \\\hline\hline
    \begin{tabular}{l}グループA\\(オーバーラップ確定提供比率)\end{tabular} & * & \begin{tabular}{l}3以上の奇数の個数の中間液滴を,\\親ノードのミキサーへ提供する\\混合ノードが含まれる場合のみ\end{tabular} \\\hline
    グループB & 3 以上  &  \\\hline
    グループC & 0 - 2 &  \\\hline
\end{tabular}
\label{table:ProvRatioGroup}
\end{table}



%\begin{itemize}
%\item 内部フラッシング(IF)の説明と,IFを起こす確率に基づく混合木内の提供比率のグループ分けの説明\cite{1}\cite{2}.
%\item 混合木を入力として受け取った直後に,その混合木に含まれるIFが必要となる可能性の高い提供比率のx2スケーリングを行う.
%\end{itemize}
\section{PMD上での大きなミキサーを用いた液滴の混合手順生成アルゴリズム}
本節では,Algorithm~\ref{alg:allscaling}の\ref{alg:samplepreparation_pseudo}~行目のPMD上での大きなミキサーを用いた液滴の混合手順生成処理について説明する.
PMD上での大きなミキサーを用いた液滴の混合手順生成処理は,大きく分けて2つの処理に分けられる.

1つ目の処理は,使用セル数推定値を用いた試薬合成におけるミキサーの配置の決定順序の並び替えである.
この処理は,使用セル数推定値という指標を用いて,試薬合成におけるミキサーの配置の決定順序の並び替えを行う.

2つ目の処理は,ミキサーの配置処理である.
使用セル数推定値を用いて並び替えた試薬合成におけるミキサーの配置の決定順序に従って,ミキサーの配置セルを決定する.
ミキサーの配置処理を進めることにより,入力された混合木に対応した試薬合成をPMD上で実行する際の液滴の混合手順が生成される.

\subsection{使用セル数推定値を用いた試薬合成における\mout{ミキサーの配置の決定順序}の並び替えアルゴリズム}
既存手法である平井らの手法では,配置優先度という指標に基づいたミキサーの配置の決定順序の並び替えによって,試薬合成におけるフラッシング回数の削減を行う.
提案手法では,スケーリングに加えて,ミキサーの配置の決定順序の並び替えも行うことで,試薬合成におけるフラッシング回数の削減を行う.
提案手法は,ミキサーの配置の決定順序の並び替えを行う際に,配置優先度に変わる新たな指標,使用セル数推定値Estimated CellUse Number(ECN)を使用する.

使用セル数推定値を用いた試薬合成におけるミキサーの配置の決定順序の並び替えアルゴリズムではまず,入力された混合木内の全ての混合ノードに使用セル数推定値を割り当てる.
混合ノードMn(nは混合ノードの番号)の使用セル数推定値は,式~\ref{equ:ECN}で計算される.
ReagentVol(Mn)は,Mnのミキサーの混合に使用される試薬液滴数を示す.
例えば,図~\ref{fig:ECNexampleTree}のM1の使用セル数推定値はECN(M1) =0+(ECN(M3)+ECN(M4)) =0+(12+2+ECN(M9)) =0+12+2+6=20と求められる.
表~\ref{table:ECNValueExample}に,図~\ref{fig:ECNexampleTree}の混合木の各混合ノードに割り当てられる使用セル数推定値の一覧を示す.
\begin{align}
    ECN(Mn)= ReagentVol(Mn) + \sum ECN(Mnの子ノード).
\label{equ:ECN}
\end{align}

\begin{figure}[tbp]
    \centering\includegraphics[scale=0.5]{img/ECNExampleTree.pdf}
 \caption{使用セル数推定値を求める混合木例}\label{fig:ECNexampleTree}
\end{figure}


使用セル数推定値の割り当て後,各混合ノードの子ノードの配置の決定順序を,使用セル数推定値の順に並び替える.
表~\ref{table:ECNValueExample}から分かるように,ECN(M2)$>$ECN(M1)である.
したがって,M0の子ノードの配置の決定はM2,M1の順で行われる.

\begin{table}[tbp]
\centering
    \caption{図~\ref{fig:ECNexampleTree}の各混合ノードに割り当てられる使用セル数推定値}
\begin{tabular}{l|r|r|r|r|r|r|r|r|r} \Hline
    &\multicolumn{1}{l|}{M1}& \multicolumn{1}{l|}{M2} & \multicolumn{1}{l|}{M3} & \multicolumn{1}{l|}{M4}& \multicolumn{1}{l|}{M5}& \multicolumn{1}{l|}{M6}&  \multicolumn{1}{l|}{M9}& \multicolumn{1}{l|}{M10}& \multicolumn{1}{l}{M11}\\\hline\hline
    使用セル数推定値& 20&22&12&8&12&10&6&6&8  \\\hline
\end{tabular}
\label{table:ECNValueExample}
\end{table}

使用セル数推定値の式~\eqref{equ:ECN}は,算出対象の混合ノードをルートとした部分木内の全ての混合ノードのミキサーでの,液滴の混合に使用されるセル数の総和の上限値を算出する.
提案手法では,使用セル数推定値を,対象の混合ノードのミキサーが液滴の混合を終えるまでに使用するセル数の概算値として用いる.
高い使用セル数推定値を割り当てられた混合ノードのミキサーが,液滴の混合を行うためには多くのセルが必要とし,他の兄弟ノードのミキサーとオーバーラップを発生させる可能性が高い.
使用セル数推定値を用いた試薬合成におけるミキサーの配置の決定順序の並び替えアルゴリズムの狙いは,ミキサーのオーバーラップを発生させる可能性が高い混合ノードほど,PMD上に空きセルが多い,小さいタイムステップで優先的に配置方法を決定することである.
そうすることで,ミキサーのオーバーラップが発生しない配置方法を選択することが容易になる.

使用セル数推定値を用いた試薬合成におけるミキサーの配置の決定順序の並び替えアルゴリズムの疑似コードを,Algorithm~\ref{alg:ECN}に示す.
\begin{algorithm}[tbp]
 \caption{使用セル数推定値を用いたミキサーの配置の決定順序の並び替え}\label{alg:ECN}
 \begin{algorithmic}[1]
     \Require $\mathit{Tree}$:2$\times$2ミキサーノードと2$\times$3ミキサーノード,試薬液滴ノードを含む混合木 

     \Function {SortByECN}{$\mathit{Tree}$}
        \State $\mathit{Children}$ = array()
       \ForAll {$\mathit{child}\gets \mathit{Tree.root.ChildrenMixingOrder}$}
            \State $\mathit{ECN}=$\Call{ ECN}{$child$} \Comment{混合木の部分木のルートである$\mathit{child}$の使用セル数推定値}
            \State Children.append(($\mathit{ECN},child$))
        \EndFor 
        \State $\mathit{SortedByECN} \gets $sorted($Children,reverse=$\True)\Comment{$\mathit{ECN}$をキー値に降順でソート}
        \State $\mathit{res}\gets$array()
        \ForAll {$\mathit{ECN,child}\gets \mathit{SortedByECN}$}
            \State $\mathit{PlacementOrderModifiedSubTree}\gets $\Call {SortByECN}{$\mathit{child}$}
            \State $\mathit{res}$.append($\mathit{PlacementOrderModifiedSubTree}$) \EndFor 
        \State $\mathit{PlacementOrderModifiedTree}\gets \mathit{Tree}$ 
        \State $\mathit{PlacementOrderModifiedTree.root.ChildrenMixingOrder}\gets \mathit{res}
        $\Return $\mathit{PlacementOrderModifiedTree}$ 
    \EndFunction 
    
     \State $\mathit{MixerVal}\gets$array($-1,\mathit{Tree.MixerNum}$)
     \Function {ECN}{$\mathit{Tree}$}
        \If {$Tree.root.isMixer == False$}\Return 0
        \Else 
            \State $\mathit{idx}\gets \mathit{Tree.root.MixerIndex}$
            \If {$\mathit{MixerVal[idx]} \geq 0$}\Return MixerVal[idx]
            \Else 
                \State  $\mathit{v}\gets 0$
                \ForAll {$\mathit{child}\gets \mathit{Tree.root.Children}$}
                    \If {$\mathit{child.isMixer} $}
                        \State $v+= $\Call{ECN}{$\mathit{child}$})
                    \Else 
                        \State $v+=\mathit{child.ProvideVolume}$\Comment{$\mathit{child}$(子ノード)が試薬液滴の場合}
                    \EndIf
                \EndFor 
                \State $\mathit{MixerVal[idx]}\gets v$
                \Return $\mathit{MixerVal[idx]}$
            \EndIf
        \EndIf
    \EndFunction 
 \end{algorithmic}
\end{algorithm}

\subsection{ミキサーの配置アルゴリズム}
本節では,PMD上での大きなミキサーを用いた液滴の混合手順生成アルゴリズムにおける2つ目の処理,ミキサーの配置について説明する.
ミキサーの配置処理では,使用セル数推定値を用いて並び替えた試薬合成におけるミキサーの配置の決定順序に従って,ミキサーの配置セルを決定する,

図~\ref{fig:GeneratingProcedure}に,ミキサーの配置処理の過程を示した.
図~\ref{fig:GeneratingProcedure}(a)は,図~\ref{fig:ECNexampleTree}と同一の混合木である.

\begin{figure}[tbp]
    \centering\includegraphics[scale=0.75]{img/GeneratingProcedure.pdf}
 \caption{ミキサーの配置過程}\label{fig:GeneratingProcedure}
\end{figure}


