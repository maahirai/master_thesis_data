\chapter{基礎知識}
本章では,本論文で扱うProgrammable Microfluidic Device(PMD)と呼ばれるバイオチップの基礎知識について説明をする.

%図は図~\ref{fig:label} のように入れます.
\section{PMDのアーキテクチャ}
%セル,バルブ,ミキサーでの液滴の混合など,PMDの基礎動作について説明する.
PMDの顕微鏡写真を図~\ref{fig:PMDMicrograph}に示す.
図~\ref{fig:PMDMicrograph}において,赤黒く見えている部品はバルブと呼ばれる.
また,バルブに囲まれた,青い線の交点となっている領域はセルと呼ばれる.
図~\ref{fig:PMDMicrograph}から分かるように,PMDはバルブに囲まれたセルが格子状に並んだ構造を持つ.

図~\ref{fig:MixingOnPMD}にPMDを用いて2種類の試薬の液滴の混合を行う過程を描いた図を示す~\cite{4}.
図~\ref{fig:MixingOnPMD}(a)は1種類目の試薬を入力ポートからPMDへと注入したことを示す図である.
図~\ref{fig:MixingOnPMD}(b)は,2種類目の試薬を入力ポートからPMDへと注入したことを示す図である.
図~\ref{fig:MixingOnPMD}(c)は,2種類の試薬の混合を行う際のPMDを示した図である.
図~\ref{fig:MixingOnPMD}(d)は,2種類の試薬の混合が行われた後のPMDを示した図である.
バルブの開閉を行うことにより,PMDは外部から注入された試薬の液滴の混合を行う.

\begin{figure}[tbp]
 \centering\includegraphics[scale=1.0]{img/PMDMicrograph.pdf}
 \caption{PMDの顕微鏡写真}\label{fig:PMDMicrograph}
\end{figure}

\begin{figure}[tbp]
    \centering\includegraphics[scale=1.0]{img/PMDMixing_jp.PDF}
    \caption{PMD 上に流し込まれた 2 種類の試薬が混合される様子,参考文献~\cite{4} より引用し一部改変}\label{fig:MixingOnPMD}
\end{figure}
\section{試薬合成}
\begin{itemize}
\item 卒論の元となった手法NTMの混合手順の図を例として,PMDを用いた試薬合成について説明する.
\item 混合木に関して,提供比率などの基礎概念について説明する.
\begin{itemize}
\item 試薬合成中に行われる液滴の混合を木構造で表したのが,混合木である...
\end{itemize} 
\end{itemize} 
\section{フラッシング}
フラッシングという操作自体と,フラッシング回数を減らしたいという研究の動機について説明する.
 
