\chapter{基礎知識}
本章では,本論文で扱うProgrammable Microfluidic Device(PMD)と呼ばれるバイオチップの基礎知識について説明をする.

%図は図~\ref{fig:label} のように入れます.
%\begin{figure}[tbp]
% \centering\includegraphics[scale=0.3]{img/org/exgate.pdf}
% \caption{caption}\label{fig:label}
%\end{figure}
\section{PMDのアーキテクチャ}
セル,バルブ,ミキサーでの液滴の混合など,PMDの基礎動作について説明する.
\section{試薬合成}
\begin{itemize}
\item 卒論の元となった手法NTMの混合手順の図を例として,PMDを用いた試薬合成について説明する.
\item 混合木に関して,提供比率などの基礎概念について説明する.
\begin{itemize}
\item 試薬合成中に行われる液滴の混合を木構造で表したのが,混合木である...
\end{itemize} 
\end{itemize} 
\section{フラッシング}
フラッシングという操作自体と,フラッシング回数を減らしたいという研究の動機について説明する.
 
