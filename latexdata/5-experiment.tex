\chapter{実験結果と考察}
本章では,提案手法であるスケーリングを用いた試薬合成におけるフラッシング回数の削減の実験の結果と,その考察を示す.

\section{評価方法}
提案手法における,スケーリングを用いた入力混合木の変形操作の評価を行うため,変形操作を行った混合木と変形操作を行わなかった混合木
のそれぞれを入力として,\rout{4章で説明した手法で}PMD上における液滴の混合手順の生成を行う.
その後,それぞれの混合木を入力として生成された混合手順におけるフラッシング回数を比較することで,
スケーリングを用いた入力混合木の変形操作によるフラッシング回数の削減率を求める.
フラッシング回数の削減率を基に,スケーリングを用いた入力混合木の変形操作の有効性を測る.

具体的な評価方法を述べる.
まず,高さ3,4,5の混合木を,300個ずつランダムに生成する.
次に,生成した全ての混合木を入力とし,PMD上における液滴の混合手順の生成を行う.
その後,生成した全ての混合木に対して,スケーリングを用いた変形操作を行う.
そして,変形操作を行った混合木を入力として,PMD上における液滴の混合手順の生成を行う.
変形操作を\rout{行った}混合木と変形操作を\rout{行わなかった}混合木の両方から生成される混合手順\rout{におけるフラッシング回数を比較し,}混合木の変形によるフラッシング回数の削減率を求める.
最後に,入力混合木の高さごとのフラッシング回数の削減率の平均を求める.
フラッシング回数の削減率の平均を基に,試薬合成におけるフラッシング回数の削減手法としての,提案手法の有効性を測る.

また,\rout{本}実験では既存手法である,配置優先度を用いた手法でもPMD上における液滴の混合手順の生成も行\rout{う}.
配置優先度を用いた手法における実験と提案手法における実験では,同じ入力混合木を使用\rout{する}.
\rout{それぞれの手法が生成する,液滴の混合手順におけるフラッシング回数を}比較することで,それぞれの手法の試薬合成におけるフラッシング回数の削減手法としての有効性を測る.

\section{実験結果}
表~\ref{table:result}に,提案手法の実験結果を示す.
また,表~\ref{table:ExistingResult}に,既存手法である配置優先度を用いた手法の実験結果を示す.
表~\ref{table:result}における変形とは,スケーリングを用いた入力混合木の変形を指す.
表~\ref{table:ExistingResult}における並び替えとは,配置優先度を用いたミキサーの配置の決定順序の並び替えを指す.
\rout{オーバーラップを発生させるミキサーの配置を行った回数やフラッシング回数などの削減率$ReductionRate$の値は,}
変形や並び替えの前の混合木を入力とした際の\rout{計算対象の}値を\rout{\it before},変形や並び替えの後の混合木を入力とした際の\rout{計算対象の}値を\rout{\it after}とすると,式\ref{equ:ReductionRate}で計算される.
ただし,式\ref{equ:ReductionRate}における割り算の結果は浮動小数点数とする.

%卒論の手法と提案手法での試薬合成におけるフラッシング回数,使用セル数を比較し,表にまとめる.
%縮小版
\begin{table}[tbp]
\centering
\newcommand{\SameWidth}[1]{\parbox[t]{6.5zw}{\raggedleft #1}}
\caption{提案手法の実験結果(小数点3桁まで)}
    \scalebox{0.6}{
\begin{tabular}{l|r|r|r|r|r|r|r} \Hline 
    \multicolumn{1}{l|}{\multirow{2}{*}{\begin{tabular}{c}入力混合木の\\高さ\end{tabular}}}& \multicolumn{1}{l|}{\multirow{2}{*}{混合ノード数}} &\multicolumn{2}{c|}{\begin{tabular}{c}オーバーラップを発生させる\\ミキサーの配置を行った回数\end{tabular}} &\multicolumn{1}{l|}{\multirow{2}{*}{\begin{tabular}{c}オーバーラップを発生させるミキサーの\\配置を行った回数の削減率の平均($\%$)\end{tabular}}}&  \multicolumn{2}{l|}{フラッシング回数} & \multicolumn{1}{l}{\multirow{2}{*}{\begin{tabular}{c}フラッシング回数の\\削減率の平均($\%$)\end{tabular}}} \\\cline{3-4}\cline{6-7}
        &&\SameWidth{変形前}&\SameWidth{変形後}&&{変形前}&{変形後}&\\\hline\hline 
        3&3.097&0.277&0.003&27.436&0.3&0.003&27.436\\\hline 
        4&6.663&1.063&0.27&53.638&1.095&0.238&59.392\\\hline 
        5&11.449&2.449&0.493&75.323&2.406&0.449&76.914\\\hline 
\end{tabular}}
\label{table:result}
\end{table}

%縮小版
\begin{table}[tbp]
\centering
\newcommand{\SameWidth}[1]{\parbox[t]{6.5zw}{\raggedleft #1}}
\caption{配置優先度を用いた手法の実験結果(小数点3桁まで)}
    \scalebox{0.58}{
\begin{tabular}{l|r|r|r|r|r|r|r} \Hline 
    \multicolumn{1}{l|}{\multirow{2}{*}{\begin{tabular}{c}入力混合木の\\高さ\end{tabular}}}& \multicolumn{1}{l|}{\multirow{2}{*}{混合ノード数}} &\multicolumn{2}{c|}{\begin{tabular}{c}オーバーラップを発生させる\\ミキサーの配置を行った回数\end{tabular}} &\multicolumn{1}{l|}{\multirow{2}{*}{\begin{tabular}{c}オーバーラップを発生させるミキサーの\\配置を行った回数の削減率の平均($\%$)\end{tabular}}}&  \multicolumn{2}{l|}{フラッシング回数} & \multicolumn{1}{l}{\multirow{2}{*}{\begin{tabular}{c}フラッシング回数の\\削減率の平均($\%$)\end{tabular}}} \\\cline{3-4}\cline{6-7}
        &&\SameWidth{並び替え前}&\SameWidth{並び替え後}&&並び替え前&並び替え後&\\\hline\hline 
        3&3.097&0.313&0.313&0&0.313&0.313&0\\\hline 
        4&6.663&1.901&1.897&0.066&1.623&1.583&0.959\\\hline 
        5&11.449&4.522&4.493&0.419&4.601&4.471&2.712\\\hline 
\end{tabular}}
\label{table:ExistingResult}
\end{table}


\begin{equation}
    ReductionRate = 
        \begin{cases}
            (1-(\rout{\it after}/\rout{\it before}))*100 & (\rout{\it before} > \rout{\it after})\\
            ((\rout{\it before}/\rout{\it after})-1)*100 & (\rout{\it before} < \rout{\it after})\\
            0 & (\rout{\it before} = \rout{\it after})
        \end{cases}
    \label{equ:ReductionRate}
\end{equation}

\rout{表~\ref{table:result},表~\ref{table:ExistingResult}の各列の値について説明する.}
混合ノード数の列は,各高さの入力混合木に含まれる混合ノード数の平均を示す.
同じ混合木を入力としたので,表~\ref{table:result},表~\ref{table:ExistingResult}において混合ノード数は同じである.
オーバーラップを発生させるミキサーの配置を行った回数の列は,液滴の混合手順において提供比率に含まれる混合ノードのミキサー同士でオーバーラップが発生する配置を行なった回数の平均を示す.
表~\ref{table:result},表~\ref{table:ExistingResult}では,配置優先度を用いた手法における高さ3の混合木を入力とした場合以外,変形後,並び替え後にこの列の値が減っている.
オーバーラップを発生させるミキサーの配置を行った回数の削減率の平均の列は,変形や並び替えによるオーバーラップを発生させるミキサーの配置を行った回数の削減率の平均を示す.
表~\ref{table:result},表~\ref{table:ExistingResult}では,配置優先度を用いた手法における高さ3の混合木を入力とした場合以外,この列の値は正の値を取っている.
これは,提案手法と配置優先度を用いた手法のどちらにおいても,オーバーラップを発生させるミキサーの配置を行った回数が削減されていることを示す.
また,入力混合木の高さが大きくなるほど,削減率の平均は高くなっている.
フラッシング回数の列は,液滴の混合手順において行なわれるフラッシング回数の平均を示す.
表~\ref{table:result},表~\ref{table:ExistingResult}では,配置優先度を用いた手法における高さ3の混合木を入力とした場合以外,変形後,並び替え後にこの列の値が減っている.
フラッシング回数の削減率の平均の列は,変形や並び替えによるフラッシング回数の削減率の平均を示す.
表~\ref{table:result},表~\ref{table:ExistingResult}では,配置優先度を用いた手法における高さ3の混合木を入力とした場合以外,この列の値は正の値を取っている.
また,入力混合木の高さが大きくなるほど,削減率の平均は高くなっている.

図~\ref{fig:FlushingReductionRate}に,提案手法と配置優先度を用いた手法における入力混合木の高さと,フラッシング回数\rout{の削減率}やオーバーラップを発生させるミキサーの配置を行なった回数の削減率の関係をグラフで示す.
図~\ref{fig:FlushingReductionRate}から,提案手法と配置優先度を用いた手法のどちらにおいても,入力混合木の高さに比例して,両方の削減率が高くなっていることが分かる.

\begin{figure}[tbp]
 \centering\includegraphics[scale=0.7]{img/bothReductionRate.pdf}

    \caption{提案手法と配置優先度を用いた手法における入力混合木の高さと,フラッシング回数\rout{の削減率や}オーバーラップを発生させるミキサーの配置を行なった回数の削減率の関係}

    \label{fig:FlushingReductionRate}
\end{figure}

\section{考察}
本節では,実験結果に対する考察を述べる.

実験結果では,配置優先度を用いた手法よりも提案手法の方が,オーバーラップを発生させるミキサーの配置を行った回数の削減率が高い.
したがって,入力混合木内のスケーリング対象提供比率を,よりミキサーのオーバーラップの起こりにくい提供比率に変更するという,スケーリングの目的が達成されていると考\rout{えられる}.

図~\ref{fig:FlushingReductionRate}から分かるように,オーバーラップを発生させるミキサーの配置を行った回数の削減率が高い入力混合木ほど,フラッシング回数の削減率が高い.
したがって,フラッシング回数の削減を行うためにはオーバーラップを発生させるミキサーの配置を行う回数を削減すれば良いという,手法の前提となっている仮説はある程度正しいことが示されたと考\rout{えられる}.

実験結果では,提案手法と配置優先度を用いた手法の両方において入力混合木の高さが大きいほど,フラッシング回数\rout{の削減率}とオーバーラップを発生させるミキサーの配置を行なった回数の削減率が高\rout{い}.
高さの大きい入力混合木は,含まれるスケーリング対象提供比率の個数も多くなりやすい.
したがって,高さの大きい入力混合木ほどスケーリングを用いた変形の影響を大きく受けるため,このような実験結果\rout{になる}と考え\rout{られる}.

