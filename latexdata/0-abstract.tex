\chapter*{内容梗概}
近年研究されているバイオチップの内の1種類に,PMD(Programmable Microfluidic Device)がある.
PMD上では,セルを環状に繋げた流路である,ミキサーを用いて液滴の混合を行うことが可能である.
PMDは,ミキサーを用いて複数種類の試薬を目標の比率で混合(試薬合成)する.

混合木と呼ばれる木構造の入力データを基に,PMD上での試薬合成を行う際に,ミキサーがセルに残した不要な中間液滴を緩衝液という液体で洗い流す(フラッシング)操作を行う場合がある.
本論文では,PMD上での試薬合成におけるフラッシング回数の削減手法である,スケーリングを用いた入力混合木の変形を提案する.
\gout{提案手法は},混合木の変形を通して試薬合成におけるフラッシング回数を削減する.

実験では,スケーリングを用いた入力混合木の変形を行った場合の試薬合成と行わなかった場合の試薬合成におけるフラッシング回数を比較した.
その結果,\gout{高さ5の混合木に対してスケーリングを用いた入力混合木の変形を行った場合,試薬合成におけるフラッシング回数が平均76.9\%削減されることを確認した.}

%ver2:ver1 で alain さんに指摘された点について修正し,その部分を赤色にしました.
%あと,アランさんにslackで既存手法に対しての新規性をはっきり書いた方が良いとアドバイスを頂いたので,その部分の文章は大幅に書き直しました.
%
%ver3:ver2に対して先生からいただいたコメントを元に修正した部分を青色にしました.
%あと,PP Valueは英語名で,配置優先度という日本語名があったことを思い出したので置換しました.
%
%ver4:2章を書きました.
%
%ver5:ver4に対してalainさんにいただいたコメントをもとに,修正した部分を緑色にしました.
%
%ver6:ver5に対して先生からいただいたコメントをもとに,修正した部分をピンク色にしました.
%    3章4章も節のタイトルを修正した方が,これからの説明の都合が良さそうなので修正し,ピンク色にしました.
%
%ver7:3章を書きました.
%
%ver8:ver7に対し,alainさんからもらった指摘をもとに赤色で修正しました.
%
%ver9:ver8に対し,山下先生からもらった指摘をもとに青色で修正しました.
%
%ver10:4章を書きました.
%
%ver11:ver10に対し,alainさんからもらった指摘をもとに緑色で修正しました.
%
%ver12:ver11に対し,先生からもらった指摘をもとにピンク色で修正しました.
%
%ver13:5章,6章を書きました.
%

