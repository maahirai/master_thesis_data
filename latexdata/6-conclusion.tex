\chapter{おわりに}
PMD は,バルブと,バルブに囲まれたセルによって構成されるデバイスである.
PMD は,バルブの開閉を行うことによりミキサーと呼ばれる環状の流路を作り,液滴の混合を行う.

ミキサーを用いた液滴の混合を複数回行うことで,PMD は複数種類の試薬を目標の比
率で混合する.この処理のことを試薬合成と呼ぶ.PMD 上での試薬合成中にはミ
キサーを用いた液滴の混合によって生成された中間液滴が,他のミキサーの使用するセル
に残され,液滴の混合を進められない状態が発生する場合がある.この状態のことをミキ
サーのオーバーラップと呼ぶ.ミキサーのオーバーラップには,残された中間液滴を緩衝
液という液体で洗い流す,フラッシングという操作を行うことで対処する.試薬合成中
に行われるフラッシング回数と比例して,試薬合成における使用緩衝液量は増加する.使
用緩衝液量の増加は実験コストの増加に繋がる.したがって,実験コストの増加を抑える
ために試薬合成中のフラッシング回数を削減することが求められている.

試薬合成中に行われるフラッシング回数を削減する手法としては,配置優先度を用いた手法が提案されている.
本論文では,PMD 上での試薬合成におけ
るフラッシング回数の削減手法,スケーリングを新たに提案する.本論文で提案する手法
のスケーリングでは,他のミキサーとのオーバーラップが発生する確率が高いと判断した
全てのミキサーの大きさが 2 倍になるように混合木を変形する.大きさを 2 倍にされたミ
キサーは,配置に使用する PMD のセル数が 2 倍になる.したがって,配置方法に様々な
選択肢を持つことが可能となり,他のミキサーとのオーバーラップが発生する配置方法を
避けることが容易になる.配置優先度を用いた手法では,配置優先度と呼ばれる指標の順
に,試薬合成中の液滴の混合順序を並び替えることによってフラッシング回数の削減を行っ
ていた.それに対して,提案手法では配置優先度を用いた手法と異なる指標の順で,試薬
合成中の液滴の混合順序の並び替えを行うのに加えて,混合木の変形も行う.これにより,
液滴の混合順序の並び替えだけでは削減することの出来なかったフラッシングも,提案手
法では削減することが可能になる.

実験では,配置優先度を用いた手法と提案手法のそれぞれを用いた試薬合成において必
要となるフラッシング回数を比較した.その結果,提案手法は配置優先度を用いた手法よりも 76.9\%少な
いフラッシング回数での試薬合成が可能であることを確認した.

実験結果より,提案手法は配置優先度を用いた手法よりも試薬合成中におけるフラッシング回数の削減効果が高いという結論を得た.

残された課題は,
提案手法は,ヒューリスティックによってミキサーの配置の決定を行っているため,ミキサーの配置の最適化アルゴリズムの開発は,これからの課題である.
また,提案手法は,スケーリングにおいて提供比率を2倍にする.
しかし,3倍や4倍など,より大きい倍率を提供比率にかけた方がよりミキサーのオーバーラップの起こりにくい提供比率に変更できる場合がある.
したがって,2倍以外の倍率でスケーリングを行う手法の開発も,これからの課題である.

%\begin{itemize}
%
%\item 本論文の概要と特徴
%\item 得られた成果
%\item それから得られる最終結論
%\item 残された課題
%    \begin{itemize}
%        \item 最適なレイアウト探索アルゴリズム
%        \item 最適なスケーリング倍率探索アルゴリズム
%    \end{itemize}
%\end{itemize}
