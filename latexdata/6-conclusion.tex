\chapter{おわりに}
PMD は,バルブと,バルブに囲まれたセルによって構成されるデバイスである.
PMD は,バルブの開閉を行うことによりミキサーと呼ばれる環状の流路を作り,液滴の混合を行う.
ミキサーを用いた液滴の混合を複数回行うことで,PMD は複数種類の試薬を目標の比
率で混合する.この処理のことを試薬合成と呼ぶ.PMD 上での試薬合成中にはミ
キサーを用いた液滴の混合によって生成された中間液滴が,他のミキサーの使用するセル
に残され,液滴の混合を進められない,ミキサーのオーバーラップという状態が発生する場合がある.
ミキサーのオーバーラップには,残された中間液滴を緩衝液という液体で洗い流す,フラッシングという操作を行うことで対処する.
試薬合成\rout{における}フラッシング回数と比例して,試薬合成における使用緩衝液量は増加する.使用緩衝液量の増加は実験コストの増加に繋がる.したがって,実験コストの増加を抑える
ために試薬合成\rout{における}フラッシング回数を削減することが求められている.
したがって,本論文では,PMD 上での試薬合成におけるフラッシング回数の削減手法,スケーリングを用いた入力混合木の変形を提案した.

\rout{本論文}では,スケーリングを用いた入力混合木の変形を行った場合の試薬合成と行わなかった場合の試薬合成において必要となるフラッシング回数を比較した.
その結果,スケーリングを用いた変形を高さ5の入力混合木に対して行った場合は,行わなかった場合よりも76.9\%少ないフラッシング回数での試薬合成が可能であることを確認した.
また,提案手法と配置優先度を用いた手法の試薬合成におけるフラッシング回数の削減効果\rout{も}比較した.
その結果,提案手法は配置優先度を用いた手法よりも試薬合成におけるフラッシング回数の削減効果が高いという結論を得た.

最後に,残された課題について述べる.
提案手法はヒューリスティックによってミキサーの配置の決定を行っている.
したがって,試薬合成におけるフラッシング回数の削減という観点において,最適なミキサーの配置を行うアルゴリズムの開発はこれからの課題である.
また,提案手法は,スケーリングにおいて提供比率を2倍にする.
しかし,3や4などより大きい倍率を提供比率に掛けた方が,よりミキサーのオーバーラップの起こりにくい提供比率に変更できる場合がある.
したがって,2以外の倍率を掛けて提供比率の変更を行うスケーリング手法の開発もこれからの課題である.
また,提案手法は表~\ref{table:ProvRatioGroup}に示す提供比率のグループに基づいて,グループAとグループBの提供比率に対してスケーリングを行う.
しかし,グループBに属する提供比率でもミキサーのオーバーラップの起こりやすさは一様でない.
混合木における提供比率の情報だけを見て,ミキサーのオーバーラップを発生させるか判定するのには限界がある.
したがって,ミキサーの配置中にオーバーラップを発生させる提供比率に対してスケーリングを動的に行うなど,
様々なタイミングでスケーリングを行うアルゴリズムの開発もこれからの課題である.
%\begin{itemize}
%
%\item 本論文の概要と特徴
%\item 得られた成果
%\item それから得られる最終結論
%\item 残された課題
%    \begin{itemize}
%        \item 最適なレイアウト探索アルゴリズム
%        \item 最適なスケーリング倍率探索アルゴリズム
%    \end{itemize}
%\end{itemize}
